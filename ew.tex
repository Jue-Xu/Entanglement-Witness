\documentclass[
%reprint,
%preprint, 
% 11pt,
%superscriptaddress,
%groupedaddress,
%unsortedaddress,
%runinaddress,
% frontmatterverbose, 
%preprintnumbers,
%nofootinbib,
%nobibnotes,
%bibnotes,
aps,
pra,
linenumbers,
% twocolumn,
% prl,
% prb,
% prd,
% rmp,
% prstab,
% prstper,
floatfix,
%longbibliography
]{revtex4-2} 
% \usepackage{revquantum}
% new linux font, ignore mono
% \usepackage[mono=false]{libertine} 
% \renewcommand{\baselinestretch}{1.05}
% \usepackage[top=0.7in,left=1in,bottom=1in,right=1in]{geometry}
\usepackage{amsmath,amsthm,amssymb,epsfig,graphicx,mathrsfs,amsfonts,dsfont,bbm}
% \usepackage{bbm} % for \mathbb{1}, but ruins the letter
% \usepackage{unicode-math}
% \DeclareMathOperator*{\argmax}{argmax}
% \DeclareMathOperator*{\argmin}{argmin}
\usepackage{pict2e}
\usepackage[percent]{overpic}
\usepackage{color}
\usepackage{listings}
\usepackage{caption}
% \usepackage{fullpage}
\usepackage[toc,title,titletoc,header]{appendix}
\usepackage{color}
\usepackage{dcolumn}
\usepackage{bm}
\usepackage{hyperref}
\hypersetup{
    citecolor=magenta,
    colorlinks=true,
    linkcolor=blue,
    filecolor=green,      
    urlcolor=cyan,
}
\usepackage[capitalise]{cleveref}
\usepackage{subcaption}
\usepackage{enumitem}
\usepackage{mathtools}
\usepackage{tikz}
%\usepackage{tikzit}
%\input{path_integral.tikzstyles}
\usepackage{braket}
\usepackage{physics}
% \usepackage{luatex85} % for qcircuit
\usepackage{luatex85,qcircuit}
\usepackage{blkarray}
\usepackage[linesnumbered,ruled,vlined,algosection]{algorithm2e}
\newcommand\mycommfont[1]{\footnotesize\ttfamily\textcolor{blue}{#1}}
\SetCommentSty{mycommfont}

% \setlength\parindent{0pt}
\setcounter{secnumdepth}{3}

\theoremstyle{plain}
\newtheorem{axiom}{Axiom}
\newtheorem{theorem}{Theorem}
\newtheorem{corollary}{Corollary}
\newtheorem{lemma}{Lemma}
\newtheorem{proposition}{Proposition}
\newtheorem{conjecture}{Conjecture} 
\newtheorem{question}{Question} 
\newtheorem{claim}{Claim} 
\theoremstyle{definition}
\newtheorem{definition}{Definition}
\newtheorem{observation}{Observation} 
\newtheorem{fact}{Fact}
\newtheorem{example}{Example}
\newtheorem{remark}{Remark}
\newtheorem{problem}{Problem}


% !TEX root = ./notes.tex

%%%%%%%%%%%%%%%%%%%%%%%%%%%%%%%%%%%%
%%%%%%%%%%%%%% math %%%%%%%%%%%%%%%%
%%%%%%%%%%%%%%%%%%%%%%%%%%%%%%%%%%%%
\newcommand{\calH}{\mathcal{calH}}
\newcommand{\hilbertspace}{\mathcal{H}}
\newcommand{\bigO}{\mathcal{O}}
\newcommand{\lagrangian}{\mathcal{L}}
\newcommand{\VS}{\textrm{VS}}

\newcommand{\realnumber}{\mathbb{R}}
\newcommand{\complexnumber}{\mathbb{C}}
\newcommand{\rationalnumber}{\mathbb{Q}}
\newcommand{\integer}{\mathbb{Z}}
\newcommand{\naturalnumber}{\mathbb{N}}
\newcommand{\numberfield}{\mathbb{F}}

\newcommand{\0}{\mathbf{0}}
\newcommand{\bI}{\mathbf{I}}
\newcommand{\identity}{\mathds{1}}
\newcommand{\midentity}{\mathds{1}}
% \newcommand{\identity}{\mathbb{1}}
\newcommand{\bX}{\mathbf{X}}
\newcommand{\bY}{\mathbf{Y}}
\newcommand{\bepsilon}{\boldsymbol{\epsilon}}

\newcommand{\ii}{\textup{i}}

\newcommand{\floor}[1]{\left\lfloor #1 \right\rfloor}
\newcommand{\ceil}[1]{\left\lceil #1 \right\rceil}

% probability
\newcommand{\probability}{\mathbb{P}}
\newcommand{\variance}{\textup{\textrm{Var}}}
\newcommand{\covariance}{\textup{\textrm{Cov}}}
\newcommand{\expectation}{\mathbb{E}}

% group theory
\newcommand{\group}{\mathbb{G}}
\newcommand{\dihedral}{\mathbb{D}}
\newcommand{\GL}{\mathbb{GL}}
\newcommand{\SL}{\mathbb{SL}}
\newcommand{\Sp}{\textup{Sp}}
% \newcommand{\sp}{\mathfrak{sp}}
\newcommand{\SU}[1]{\textup{SU(#1)}}
\newcommand{\su}[1]{\mathfrak{su}(#1)}
% \renewcommand{\SO}[1]{\textup{SO(#1)}}
% \newcommand{\SO}{\textup{SO}}

% graph theory
\newcommand{\graph}{G}

% matrix and linear algebra
\newcommand{\diag}{\textup{diag}}
% \let\span\relax
% \DeclareMathOperator{\span}{\textup{span}}
% \newcommand{\span}{\textup{span}}
\newcommand{\spn}{\mathop{\mathrm{span}}}
\DeclareMathOperator{\spann}{\textup{span}}
%%%%%%%%%%%%%%%%%%%%%%%%%%%%%%%%%%%%
%%%%%%%%%%%%%%  CS  %%%%%%%%%%%%%%%%
%%%%%%%%%%%%%%%%%%%%%%%%%%%%%%%%%%%%
% cryptography
\newcommand{\gen}{\textsf{Gen}}
\newcommand{\enc}{\textsf{Enc}}
\newcommand{\dec}{\textsf{Dec}}
\newcommand{\mac}{\textsf{Mac}}
\newcommand{\sign}{\textsf{Sign}}
\newcommand{\verfy}{\textsf{Verfy}}
\newcommand{\negl}{\textup{negl}}

% quantum computing
% gates
\newcommand{\cnot}{\textup{\textsc{cnot}}}
\newcommand{\hdm}{\textup{\textsc{h}}}
\newcommand{\tphase}{\textup{\textsc{t}}}
\newcommand{\cphase}{\textup{\textsc{cphase}}}
\newcommand{\swap}{\textup{\textsc{swap}}}
\newcommand{\negate}{\textup{\textsc{not}}}
\newcommand{\QFT}{\textup{QFT}}

% Boolean Functions
\newcommand{\MAJ}{\textup{\textsc{maj}}}
\newcommand{\NOT}{\textup{\textsc{not}}}
\newcommand{\OR}{\textup{\textsc{or}}}
\newcommand{\AND}{\textup{\textsc{and}}}
\newcommand{\NAND}{\textup{\textsc{nand}}}
\newcommand{\EQ}{\textup{\textsc{eq}}}
\newcommand{\IP}{\textup{\textsc{ip}}}
\newcommand{\DISJ}{\textup{\textsc{disj}}}
\newcommand{\Parity}{\textup{\textsc{parity}}}
\newcommand{\Threshold}{\textup{\textsc{thr}}}

\newcommand{\GS}{\textup{\textsc{gs}}}
\newcommand{\dejo}{\textup{\textsc{DeJo}}}
\newcommand{\STAB}{\textup{\textsc{stab}}}

% algorithms
\newcommand{\algo}{\mathcal{A}}
\newcommand{\maxcut}{\textup{\textsc{MaxCut}}}
\newcommand{\sat}{\textup{\textsc{sat}}}
\newcommand{\partition}{\textup{\textsc{Partition}}}
\newcommand{\bosonsample}{\textup{\textsc{BosonSampling}}}

% complexity measures
\newcommand{\vcdim}{\mathsf{VCdim}}
\DeclareMathOperator{\certificate}{\mathsf{Cert}}
\DeclareMathOperator{\s}{\mathsf{s}}
\DeclareMathOperator{\bs}{\mathsf{bs}}
\DeclareMathOperator{\adeg}{\mathsf{\widetilde{deg}}}
% \DeclareMathOperator{\adv}{\mathsf{Adv}}
\DeclareMathOperator{\dqc}{\mathsf{D}}
\DeclareMathOperator{\rqc}{\mathsf{R}}
\DeclareMathOperator{\qqc}{\mathsf{Q}}
\DeclareMathOperator{\cmc}{\mathsf{C}}
\DeclareMathOperator{\rcmc}{\mathsf{RC}}
\DeclareMathOperator{\qcmc}{\mathsf{QC}}
\let\deg\relax
\DeclareMathOperator{\deg}{\mathsf{deg}}
\DeclareMathOperator{\poly}{\textup{poly}}

% complexity classes
\newcommand{\reduceto}{\le_P}
\let\cclass\textup
\let\P\relax
\newcommand{\P}{\cclass{P}}
\newcommand{\PP}{\cclass{PP}}
\newcommand{\NP}{\cclass{NP}}
\newcommand{\sharpP}{\cclass{\#P}}
\newcommand{\coNP}{\cclass{co-NP}}
\newcommand{\PH}{\cclass{PH}}
\newcommand{\NPC}{\cclass{NPC}}
\newcommand{\BQP}{\cclass{BQP}}
\newcommand{\QMA}{\cclass{QMA}}
\newcommand{\PSPACE}{\cclass{PSPACE}}
\newcommand{\BPP}{\cclass{BPP}}

% Optimization
\newcommand{\subjectto}{\textup{subject to  }}

\let\iff\relax
\newcommand{\iff}{\text{iff}}
\newcommand{\eff}{\textup{eff}}
\newcommand{\st}{\text{ s.t. }}
\newcommand{\otherwise}{\text{otherwise}}
\newcommand{\T}{\intercal}
\newcommand{\OPT}{\textup{OPT}}


\newcommand\vartextvisiblespace[1][.5em]{%
  \makebox[#1]{%
    \kern.07em
    \vrule height.3ex
    \hrulefill
    \vrule height.3ex
    \kern.07em
  }% <-- don't forget this one!
}
\newcommand{\visiblespace}{\vartextvisiblespace}

%%%%%%%%%%%%%%%%%%%%%%%%%%%%%%%%%%%%
%%%%%%%%%%%%% Physics %%%%%%%%%%%%%%
%%%%%%%%%%%%%%%%%%%%%%%%%%%%%%%%%%%%
\newcommand{\zpartition}{\mathcal{Z}}
\newcommand{\llaplacian}{\mathfrak{L}}
\newcommand{\dlagrangian}{\mathcal{L}}
\newcommand{\eaction}{\mathcal{A}}
\newcommand{\action}{\mathcal{S}}
\newcommand{\hhat}{\hat{H}}
\newcommand{\xhat}{\hat{x}}
\newcommand{\phat}{\hat{p}}
\newcommand{\qhat}{\hat{q}}
\newcommand{\nhat}{\hat{n}}
\newcommand{\pihat}{\hat{\pi}}
\newcommand{\phihat}{\hat{\phi}}
\newcommand{\oph}{\mathbf{H}}
\newcommand{\opx}{\mathbf{x}}
\newcommand{\opp}{\mathbf{p}}
\newcommand{\opq}{\mathbf{q}}
\newcommand{\vecx}{\vec{x}}
\newcommand{\vecp}{\vec{p}}
\newcommand{\veck}{\vec{k}}
\newcommand{\vecq}{\vec{q}}
\newcommand{\vbk}{\vb{k}}
\newcommand{\vbs}{\vb{s}}
\newcommand{\vbx}{\vb{x}}
\newcommand{\vbn}{\vb{n}}
\newcommand{\vbp}{\vb{p}}
\newcommand{\vbq}{\vb{q}}
\newcommand{\vbr}{\vb{r}}
\newcommand{\vbe}{\vb{e}}
\newcommand{\vbv}{\vb{v}}
\newcommand{\vbw}{\vb{w}}
\newcommand{\vbB}{\vb{B}}
\newcommand{\vbE}{\vb{E}}
% \newcommand{\acreation}{\hat{a}^\dagger}
% \newcommand{\aannihilation}{\hat{a}}
\newcommand{\acreation}{\hat{a}^\dagger}
\newcommand{\aannihilation}{\hat{a}}
\newcommand{\bcreation}{\hat{b}^\dagger}
\newcommand{\bannihilation}{\hat{b}}
\newcommand{\ccreation}{\hat{c}^\dagger}
\newcommand{\cannihilation}{\hat{c}}
\newcommand{\homega}{\hbar \omega}
\newcommand{\opsigma}{\hat{\bm{\sigma}}}
\newcommand{\hatsigma}{\hat{\sigma}}
\newcommand{\bmhsig}{\bm{\hat{\sigma}}}
\newcommand{\hsig}{\hat{\sigma}}
\newcommand{\si}{\hat{\sigma}_0}
\newcommand{\sx}{\hat{\sigma}_x}
\newcommand{\sy}{\hat{\sigma}_y}
\newcommand{\sz}{\hat{\sigma}_z}
\newcommand{\splus}{\hat{\sigma}_+}
\newcommand{\sminus}{\hat{\sigma}_-}
\newcommand{\px}{\hat{X}}
\newcommand{\py}{\hat{Y}}
\newcommand{\pz}{\hat{Z}}
\newcommand{\pI}{\hat{I}}
\newcommand{\schrodinger}{\textup{Schr\"{o}dinger }}
\newcommand{\tc}{T_c}
\newcommand{\alembertian}{\square}
\newcommand{\vecA}{\vb{A}}
\newcommand{\magfield}{\vb{B}}
\newcommand{\elefield}{\vb{E}}

\newcommand{\deltat}{\Delta t}
\newcommand{\deltatau}{\Delta \tau}

%%%%%%%%%%%%%%%%%%%%%%%%%%%%%%%%%%%%
%%%%%%%%%%%%% Quantum Computing %%%%%%%%%%%%%%
%%%%%%%%%%%%%%%%%%%%%%%%%%%%%%%%%%%%
% \newcommand{\gcommutator}[1]{[[ #1 ]]}
\usepackage{stmaryrd}
\newcommand{\gcommutator}[1]{\llbracket #1 \rrbracket}
\renewcommand{\llaplacian}{\hat{\mathfrak{L}}}
%\newcommand{\zpartition}{\mathcal{Z}}
\newcommand{\hamiltonian}{\hat{H}}
\newcommand{\ew}{\hat{W}}
\newcommand{\ob}{\hat{O}}
\newcommand{\U}{\hat{U}}
\newcommand{\dm}{\hat{\rho}}
\newcommand{\oracle}{\hat{O}}
\newcommand{\D}{\mathcal{D}}
\newcommand{\proj}{\hat{P}}
%\newcommand{\deltat}{\Delta t}
%\newcommand{\deltatau}{\Delta \tau}
\newcommand{\cz}{\textup{\textsc{cz}}}
\newcommand{\cx}{\textup{\textsc{cx}}}
\newcommand{\toffoli}{\textup{\textsc{toffoli}}}
\newcommand{\lleft}{\leftarrow}
\newcommand{\rright}{\rightarrow}
\newcommand{\intinf}{\int_{-\infty}^{\infty}}

% disable subsections and subsubsections in the TOC
\makeatletter
%\def\l@subsection#1#2{}
\def\l@subsubsection#1#2{}
\makeatother

\begin{document}
%%%%%%%%%%%%%%%%%%%
\title{Entanglement witness by quantum circuits}
\author{Jue Xu}
\email{juexu@cs.umd.edu}
\author{Qi Zhao}
\email{email}
% \affiliation{Department of Computer Science, University of Maryland, College Park.}
\date{\today}
%%%%%%%%%%%%%%%%%%%
% \vspace{10mm}
\begin{abstract}
	Machine learning algorithms are applied to the entanglement detection problem.
	We compare complexity and performance of different methods, including conventional methods, machine learning, quantum algorithms.
\end{abstract}

\maketitle
% \setcounter{tocdepth}{0}
 \tableofcontents
% \newpage

%%%%%%%%%%%%%%%Content%%%%%%%%%%%%%%%
\section{Introduction}
Entanglement \cite{horodeckiQuantumEntanglement2009} is the key ingredient of quantum computation \cite{}, quantum communication \cite{}, and quantum cryptography \cite{}.
It is essential to benckmark (characterize) entanglement structures of target states.

\section{Preliminary}
% \subsection{Related works}
\subsection{Notations}
The (classical) training data is a set of $m$ data points $\qty{(\vbx^{(i)}, y^{(i)})}^{m}_{i=1}$ 
where each data point is a pair $(\vbx,y)$.
Normally, the input $\vbx:= (x_1,x_2,\dots,x_d) \in \realnumber^d$  is a vector where $d$ is the number of \emph{features}
and its \emph{label} $y\in\Sigma$ is a scalar with some discrete set $\Sigma$ of alphabet/categories. 
For simplicity, we assume $\Sigma=\qty{-1,1}$ (binary classification).

Notations: a graph $\graph=(V,E)$ with vertices $V$ and edges $E$; 
% a group $\group$ with a subgroup $\subgroup$. 
The hats on the matrices such as $\hat{A}$, $\hamiltonian$, $\dm$, $\ob$, $\ew$, emphasize that they play the roles of operators.
denote vector (matrix) $\vbx$, $\vb{K}$ by boldface font.

For specific purpose, we use different basis (representations) for quantum states.
One is the computational basis $\qty{\ket{z}}$ with $z\in \qty[2^n]$ where $n$ is the number of qubits,
while the other useful one is the binary representation of computational basis $\qty{\ket{\vbx}\equiv\ket{x_1,x_2,\dots,x_n}}$ with $x_j\in \qty{0,1}$. 
For simplicity, we let $N \equiv 2^n$ and $\ket{\vb{0}}\equiv\ket{0^n} \equiv\ket{0}^{\otimes n}$ if no ambiguity.
$\ket{+}: = (\ket{0}+\ket{1})/\sqrt{2} $

\subsection{Entanglement detection}
\begin{definition}[Entangled state]
	pure state; mixed state is convex combination of entangled ...
\end{definition}
\begin{definition}[Bipartite state]
\end{definition}

\subsubsection{entanglement structures}
Given a $n$-partite quantum system and its partition into $m$ subsystems, the \emph{entanglement structure} indicates how the subsystems are entangled with each other.
In some specific systems, such as distributed quantum computing[] quantum networks[] or atoms in a lattice, the geometric configuration can naturally determine the system partition.
\begin{remark}
	Compared with genuine entanglement, multipartite entanglement structure still lacks a systematic exploration, due to the rich and complex structures of N-partite system.
	Unfortunately, it remains an open problem of efficient entanglement-structure detection of general multipartite quantum states.
\end{remark}
\begin{definition}[Multi-partite state]
\end{definition}
\begin{definition}[Fully separable state?]\label{def:fully_separable}
	An $n$-qubit pure state $\ket{\psi_f}$ is \emph{P-fully separable} \iff it can be written as 
	$\ket{\psi_f}=\otimes_i^m \ket{\phi_{A_i}}$.
	An $n$-qubit mixed state $\dm_f$ is P-fully separable $\iff$ it can be decomposed into a conex mixture of P-fully separable pure states 
	\begin{equation}
		\dm_f = \sum_i p_i \op{\psi_f^i}, (\forall i) ( p_i\ge 0, \sum_i p_i = 1) .
	\end{equation}
	P-bi-separable... $S_f^P \subset S_b^P$
\end{definition}
By going through all possible partitions, one can investigate higher level entanglement structures, such as entanglement intactness (non-separability), which quantifies how many pieces in the $n$-partite state are separated.
\begin{definition}[Entanglement intactness, depth]
	the entanglement intactness of a state $\dm$ to be $m$, \iff $\dm\notin S_{m+1}$ and $\dm\in S_m$.
\end{definition}
\begin{remark}
	When the entanglement intactness is 1, the state is \nameref{def:genuinely_entangled}; and when the intactness is N, the state is fully separable.
\end{remark}
\begin{example}[GHZ, W]
	bipartite: Bell states;
	nontrivial multipartite: tripartite
\end{example}

\begin{definition}[genuine entangled]\label{def:genuinely_entangled}
	A state $\dm$ possesses P-genuine entanglement iff $\dm\notin S_b^P$.
	A state possesses P-genuine entanglement if it is outside of $S_b^P$.
	A state possesses \emph{genuine multipartite entanglement} (GME) if it is outside of $S_2$, and is (fully) $n$-separable if it is in $S_n$.
\end{definition}
\begin{figure}[!ht]
	\centering
	\begin{subfigure}{0.3\textwidth}
		\centering
		\includegraphics[width=.9\linewidth]{gme.png}
	\end{subfigure}
	\begin{subfigure}{0.3\textwidth}
		\centering
		\includegraphics[width=.8\linewidth]{sep.png}
	\end{subfigure}
	\begin{subfigure}{0.3\textwidth}
		\centering
		\includegraphics[width=.9\linewidth]{ppt.png}
	\end{subfigure}
	\caption{(a) entanglement witness, PPT criteria, SVM (kernel)?. (c) convex hull... }
	\label{fig:entangle}
\end{figure}

\subsubsection{Graph state}
graph state is an important class of multipartite states in quantum information.
cluster state is the special case of graph state.
2D cluster state is the universal resource for the measurement based quantum computation (MBQC) \cite{briegelMeasurementbasedQuantumComputation2009}.
\begin{definition}[graph state]
	Given a graph $G=(V,E)$, a graph state is constructed as 
	\begin{itemize}
		\item vertices: $\ket{+}^{\otimes n}$
		\item edges: apply controlled-Z to every edge,
		that is $\ket{G}=\prod_{(i,j)\in E}\textsf{cZ}_{(i,j)} \ket{+}^{\otimes n}$
	\end{itemize}
	An $n$-partite graph state can also be uniquely determined by $n$ independent stabilizers, 
	$S_i:= X_i \bigotimes_{j\in n}Z_j$, 
	which commute with each other and $\forall i,S_i\ket{G}=\ket{G}$.
\end{definition}
\begin{problem}[Certify entanglement]
	Multipartite entanglement-structure detection
	\begin{itemize}
		\item \textbf{Input}: Given a state close to a \textbf{known} state $\ket{\psi}$,
		\item \textbf{Output}: the certified lower-order entanglement among several subsystems could be still useful for some quantum information tasks.
		entanglement structure
	\end{itemize}
\end{problem}
\begin{remark}
	The graph state is the unique eigenstate wtih eignevalue of +1 for all the $n$ stabilizers.
	As a result, a graph state can be writteb as a product of stailizer projectors, $\op{G}=\prod_{i=1}^n \frac{S_i +\identity}{2}$.
	stabilizer formalism?; 
\end{remark}
\begin{remark}
	The entanglement entropy $S( \dm_A )$ equals the rank of the adjacency matrix of the underlying bipartite graph, which can be efficiently calculated.
\end{remark}
\begin{proposition}[\cite{zhouDetectingMultipartiteEntanglement2019}]
	Given a graph state $\ket{G}$ and a partition $\mathcal{P}=\qty{A_i}$, the fidelity between $\ket{G}$ and any \nameref{def:fully_separable} is upper bounded by
	\begin{equation}
		\Tr(\op{G} \dm_f) \le \min_{\qty{A,\bar{A}}} 2^{-S(\dm_A)}
	\end{equation}
	where $S(\dm_A)$ is the von Neumann entropy of the reduced density matrix $\dm_A=\Tr_{\bar{A}}(\op{G})$.
\end{proposition}
\begin{theorem}
	k local measurements. Here, k is the chromatic number of the corresponding graph, typically, a small constant independent of the number of qubits.
\end{theorem}
\begin{proposition}[Entanglement of graph state]
	\cite{heinEntanglementGraphStates2006}.
	witness; bounds
\end{proposition}

\subsubsection{Entanglement witness}
\begin{theorem}[\cite{gurvitsClassicalDeterministicComplexity2003}]
	% The problem of determining whether a given quantum state is entangled lies at the heart of quantum information processing, which is an NP-hard problem in general.
	The weak membership problem for the convex set of separable normalized bipartite density matrices is NP-Hard.
	\begin{itemize}
		\item \textbf{Input}: ??
		\item \textbf{Output}: ??
	\end{itemize}
\end{theorem}
\begin{question}
	specific cases? approximately correct? quantum computation? machine learning (data)?
\end{question}
\begin{theorem}[PPT criterion]
	the positive partial transpose (PPT) criterion, saying that a separable state must have PPT.
	Note, it is only necessary and sufficient when $d_A d_B \le 6$.
\end{theorem}
see \cref{fig:entangle}
\begin{definition}[Entanglement witness]
	entanglement witness $\ew$
	\begin{equation}
		\Tr(\ew\dm) \ge 0 , \forall \text{ separable };\quad
		\Tr(\ew\dm) < 0 , \text{ for some entangled }
	\end{equation}
\end{definition}
natural question: nonlinear EW \cite{guhneNonlinearEntanglementWitnesses2006} (kernel method)
\begin{definition}[Fidelity]\label{def:fidelity}
	Given a pair of states (target and real), 
	\begin{equation}
		F(\ket{\psi},\ket{\psi'}) :=
	\end{equation}
\end{definition}
\begin{problem}[Fidelity estimate]
	defined as follows
	\begin{itemize}
		\item \textbf{Input}: Given two density matrices $\dm$ and $\dm'$, 
		\item \textbf{Output}: \nameref{def:fidelity} with error $\epsilon$
	\end{itemize}
\end{problem}
\begin{problem}[Trace/expectaton estimate]
	defined as follows
	\begin{itemize}
		\item \textbf{Input}: Given an observable $\ob$ and a mixed state $\dm$ in density matrix,
		\item \textbf{Output}: the expectation value $\Tr(\ob \dm) = $ with error $\epsilon$ (trace distance)
	\end{itemize}
\end{problem}
\begin{problem}[Entanglement witness with prior]
	% \cite{zhouDetectingMultipartiteEntanglement2019}
	with prior knowledge
	\begin{itemize}
		\item \textbf{Input}: a \textbf{known} state $\ket{\psi}$, with noise
		\item \textbf{Output}: ???
	\end{itemize}
	decision problem
\end{problem}

\subsection{Shadow tomography}
Intuitively, a general tomography \cite{altepeterPhotonicStateTomography2005} that extract all information about a state requires exponential copies (samples/measurements).
Inspired by Aaronson's shadow tomography \cite{aaronsonShadowTomographyQuantum2018}, Huang et. al \cite{huangPredictingManyProperties2020}
\begin{problem}[Shadow tomography]
	\emph{shadow tomography}
	\begin{itemize}
		\item \textbf{Given (Input):} \textbf{unknown} $D$-dimensional mixed state $\rho$, known 2-outcome measurements $E_1,\dots,E_M$
		\item \textbf{Goal (Output):} estimate $\probability[E_i \text{ accept } \dm]$ to within additive error $\epsilon$, $\forall i\in [M]$, with $\ge 2/3$ success probability
	\end{itemize}
\end{problem}
\begin{theorem}[\cite{aaronsonShadowTomographyQuantum2018}]
	It is possible to do shadow tomography using $\tilde{\bigO}(\frac{\log^4 M\cdot \log D}{\epsilon^4})$ copies. [no construction algorithm?]
	sample complexity lower bound $\Omega(\log M\cdot \epsilon^{-2})$, 
\end{theorem}
	random Pauli measurements
\begin{definition}[classical shadow]\label{def:classical_shadow}
	classical shadow
	\begin{equation}
		\dm_{cs} = \mathcal{M}^{-1} \qty(U^\dagger \op{\hat{b}} U)
	\end{equation}
\end{definition}
predict linear function with classical shadows
\begin{equation}
	o_i = \Tr(O_i \dm_{cs})
	\text{ obeys }
	\expectation[o] =\Tr(O_i \dm)
\end{equation}
\begin{lemma}
	the variance
	\begin{equation}
		\variance[o] = \expectation[(o-\expectation[o])^2]
		\le \norm{O - \frac{\Tr(O)}{2^n} \identity}^2_{\text{shadow}}
	\end{equation}
\end{lemma}
sample complexity
\begin{equation}
	N_{tot} = \bigO \qty(
		\frac{\log (M)}{\epsilon^2} \max_{1\le i\le M} 
		\norm{O_i - \frac{\Tr(O_i)}{2^n} \identity}^2_{shadow}
	)
\end{equation}
\begin{theorem}[Pauli/Clifford measurements]
	additive error $\epsilon$, $M$ arbitrary $k$-local linear function $\Tr(\ob_i\dm)$,
	% lower bound
	$\Omega(\log(M) 3^k/\epsilon^2)$ copies of the state $\dm$.
\end{theorem}

\section{Classical and Quantum Algorithms}
We consider the problem 
\begin{problem}[???]
	problem without training data
	\begin{itemize}
		\item \textbf{Input}: a graph $\graph$ encoding in a graph state $\ket{\graph}$
		\item \textbf{Output}: entanglement structure
	\end{itemize}
	with training data
	\begin{itemize}
		\item \textbf{features}:
		\item label: 
	\end{itemize}
\end{problem}

\subsection{Quantum-classical (ML) hybrid method}
\subsubsection{Classical machine learning}
separability classifier by neural network \cite{luSeparabilityEntanglementClassifierMachine2018}.
rigorous quantum advantage of quantum kernel method in SVM \cite{liuRigorousRobustQuantum2021}.
classical machine learning with \nameref{def:classical_shadow} \cite{huangProvablyEfficientMachine2021}.
\begin{definition}[SVM]
	find a hyperplane (a linear function)
\end{definition}
nonlinear boundary. map to a higher dimensional (feature) space, in which data is linearly separable.
\begin{definition}[Kernel method]
	Gaussian kernel; 
	graph kernel;
	shadow kernel
\end{definition}
\begin{figure}[!ht]
	\centering
	\includegraphics[width=.35\linewidth]{data.png}
	\caption{computational model powered by training data}
\end{figure}
\begin{theorem}[power of data]
	data learning
\end{theorem}

\subsubsection{Quantum trace estimation}
The task of estimating quantities like 
\begin{equation}
	\Tr(\rho_1 \cdots \rho_m)
	\tag{multivariate traces}
\end{equation}
given access to copies of the quantum states $\rho_1$  through $\rho_m$.
% is a fundamental building block in quantum information science
\begin{theorem}[Quantum trace estimation]
	multivariate trace estimation can be implemented in constant quantum depth, with only linearly-many controlled two-qubit gates and a linear amount of classical pre-processing	
\end{theorem}

\subsection{Variational quantum circuits}
\subsubsection{Variational quantum kernel estimation}
an ansatz
\begin{equation}
	\ew_{a} := \sum_i  a_{..} \bigotimes \hat{\sigma}^{(n)}
	,\quad \hat{\sigma} \in \qty{\sx,\sy,\sz,I}
\end{equation}
\begin{algorithm}[H]
    \DontPrintSemicolon
    \SetKwInOut{Input}{input}
    \SetKwInOut{Output}{output}
    \Input{density matrix $\dm$}
    \Output{determine entangled structure??}
    \BlankLine
    \For{ $i = 1,2, \ldots, m$} {
        $W_i$  \tcp*{this is a comment}
        % \tcc{comment in a new line}
    {\Return "separable?"}
    }
    \Return entangled ?
    \caption{Entanglement witness by ...}
    \label{alg:miller_rabin}
\end{algorithm}

\subsubsection{Variational trace estimate}
find optimal entanglement witness (qunatum circuit?)

\subsection{Theoretical upper bounds and lower bounds}
\begin{definition}[graph property]\label{def:graph_property}
	monotone
\end{definition}
\begin{problem}[Graph property test]
\end{problem}
quantum advantage
\begin{itemize}
	\item no input encoding problem \cite{tangQuantumPrincipalComponent2021}
	\item contrived problem? for exponential speedup
	\item convex body query? complexity
\end{itemize}
\begin{table}[!ht]
\centering
\begin{tabular}{c|c|c|c}
	& gate/depth/computation & query?complexity & measurements/samples \\  
	\hline
	shadow tomography:  & & & $\bigO$, $\Omega$ \\  
	entanglement witness (no ML, data); & & & \\  
	classical machine learning;  & & & \\  
	quantum (variational) circuits & & & \\  
	\hline
\end{tabular}
\caption{complexity measures of different methods}
\end{table}

\section{Numerical Simulation}
\subsection{Classification accuracy}
performance of different methods: 
% \begin{itemize}
% 	\item shadow tomography: 
% 	\item entanglement witness (no machine learning); 
% 	\item classical machine learning; 
% 	\item quantum machine learning
% \end{itemize}

\subsection{Robustness to noise}
tradeoff between (white noise) tolerance (robustness) and efficiency (number of measurements).
\begin{equation}
	\dm_{noise} = (1-p_{noise}) \op{G} + p_{noise} \frac{\identity}{2^n}
\end{equation}
$p_{noise}$ indicates the robustness of the algorithm (witness).
\begin{remark}
	the largest noise tolerance $p_{limit}$ just related to the \textbf{chromatic number} of the graph.[??]
	\nameref{def:graph_property}
\end{remark}
% \input{complexity.tex}
% \input{optimization.tex}

\section{Conclusion and Discussion}
todo
\begin{itemize}
	\item experiment (generation, verification) \cite{luEntanglementStructureEntanglement2018}
	\item error correction?
\end{itemize}

\subsection*{Acknowledgements}
% \thanks{The author thanks} 
% The author thanks
% TikZiT, QuTip

%\begin{appendices}
    %\chapter{}
%\end{appendices}

% %%%%%%%%%%%%%%%Reference%%%%%%%%%%%%%%%
% % \newpage
% % \printbibliography
\bibliographystyle{apsrev4-2}
%\bibliographystyle{alpha}
\bibliography{ref}

%\begin{widetext}
\onecolumngrid
\appendix

\section{Machine learning background}
% In this work, we restrict ourself to supervised learning (mainly SVM), where we are given a set of labeled data for training to predict labels of new data.
% \subsection{SVM}
% \subsection{neural network}
% \subsection{Unsupervised: PCA}

%\end{widetext}

\end{document}