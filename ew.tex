\documentclass[
%reprint,
%preprint, 
%11pt,
%superscriptaddress,
%groupedaddress,
%unsortedaddress,
%runinaddress,
% frontmatterverbose, 
%preprintnumbers,
%nofootinbib,
%nobibnotes,
%bibnotes,
aps,
pra,
linenumbers,
% twocolumn,
% prl,
% prb,
% prd,
% rmp,
% prstab,
% prstper,
floatfix,
%longbibliography
]{revtex4-2} 
% \usepackage{revquantum}
% new linux font, ignore mono
% \usepackage[mono=false]{libertine} 
% \renewcommand{\baselinestretch}{1.05}
% \usepackage[top=0.7in,left=1in,bottom=1in,right=1in]{geometry}
\usepackage{amsmath,amsthm,amssymb,epsfig,graphicx,mathrsfs,amsfonts,dsfont,bbm}
% \usepackage{bbm} % for \mathbb{1}, but ruins the letter
% \usepackage{unicode-math}
% \DeclareMathOperator*{\argmax}{argmax}
% \DeclareMathOperator*{\argmin}{argmin}
\usepackage{pict2e}
\usepackage[percent]{overpic}
\usepackage{color}
\usepackage{listings}
\usepackage{caption}
% \usepackage{fullpage}
\usepackage[toc,title,titletoc,header]{appendix}
\usepackage{color}
\usepackage{dcolumn}
\usepackage{bm}
\usepackage{hyperref}
\hypersetup{
    citecolor=magenta,
    colorlinks=true,
    linkcolor=blue,
    filecolor=green,      
    urlcolor=cyan,
}
\usepackage[capitalise]{cleveref}
\usepackage{subcaption}
\usepackage{enumitem}
\usepackage{mathtools}
\usepackage{tikz}
%\usepackage{tikzit}
%\input{path_integral.tikzstyles}
\usepackage{braket}
\usepackage{physics}
% \usepackage{luatex85} % for qcircuit
\usepackage{luatex85,qcircuit}
\usepackage{blkarray}

% \setlength\parindent{0pt}
\setcounter{secnumdepth}{3}

\theoremstyle{plain}
\newtheorem{axiom}{Axiom}
\newtheorem{theorem}{Theorem}
\newtheorem{corollary}{Corollary}
\newtheorem{lemma}{Lemma}
\newtheorem{proposition}{Proposition}
\newtheorem{conjecture}{Conjecture} 
\newtheorem{question}{Question} 
\newtheorem{claim}{Claim} 
\theoremstyle{definition}
\newtheorem{definition}{Definition}
\newtheorem{observation}{Observation} 
\newtheorem{fact}{Fact}
\newtheorem{example}{Example}
\newtheorem{remark}{Remark}
\newtheorem{problem}{Problem}


% !TEX root = ./notes.tex

%%%%%%%%%%%%%%%%%%%%%%%%%%%%%%%%%%%%
%%%%%%%%%%%%%% math %%%%%%%%%%%%%%%%
%%%%%%%%%%%%%%%%%%%%%%%%%%%%%%%%%%%%
\newcommand{\calH}{\mathcal{calH}}
\newcommand{\hilbertspace}{\mathcal{H}}
\newcommand{\bigO}{\mathcal{O}}
\newcommand{\lagrangian}{\mathcal{L}}
\newcommand{\VS}{\textrm{VS}}

\newcommand{\realnumber}{\mathbb{R}}
\newcommand{\complexnumber}{\mathbb{C}}
\newcommand{\rationalnumber}{\mathbb{Q}}
\newcommand{\integer}{\mathbb{Z}}
\newcommand{\naturalnumber}{\mathbb{N}}
\newcommand{\numberfield}{\mathbb{F}}

\newcommand{\0}{\mathbf{0}}
\newcommand{\bI}{\mathbf{I}}
\newcommand{\identity}{\mathds{1}}
\newcommand{\midentity}{\mathds{1}}
% \newcommand{\identity}{\mathbb{1}}
\newcommand{\bX}{\mathbf{X}}
\newcommand{\bY}{\mathbf{Y}}
\newcommand{\bepsilon}{\boldsymbol{\epsilon}}

\newcommand{\ii}{\textup{i}}

\newcommand{\floor}[1]{\left\lfloor #1 \right\rfloor}
\newcommand{\ceil}[1]{\left\lceil #1 \right\rceil}

% probability
\newcommand{\probability}{\mathbb{P}}
\newcommand{\variance}{\textup{\textrm{Var}}}
\newcommand{\covariance}{\textup{\textrm{Cov}}}
\newcommand{\expectation}{\mathbb{E}}

% group theory
\newcommand{\group}{\mathbb{G}}
\newcommand{\dihedral}{\mathbb{D}}
\newcommand{\GL}{\mathbb{GL}}
\newcommand{\SL}{\mathbb{SL}}
\newcommand{\Sp}{\textup{Sp}}
% \newcommand{\sp}{\mathfrak{sp}}
\newcommand{\SU}[1]{\textup{SU(#1)}}
\newcommand{\su}[1]{\mathfrak{su}(#1)}
% \renewcommand{\SO}[1]{\textup{SO(#1)}}
% \newcommand{\SO}{\textup{SO}}

% graph theory
\newcommand{\graph}{G}

% matrix and linear algebra
\newcommand{\diag}{\textup{diag}}
% \let\span\relax
% \DeclareMathOperator{\span}{\textup{span}}
% \newcommand{\span}{\textup{span}}
\newcommand{\spn}{\mathop{\mathrm{span}}}
\DeclareMathOperator{\spann}{\textup{span}}
%%%%%%%%%%%%%%%%%%%%%%%%%%%%%%%%%%%%
%%%%%%%%%%%%%%  CS  %%%%%%%%%%%%%%%%
%%%%%%%%%%%%%%%%%%%%%%%%%%%%%%%%%%%%
% cryptography
\newcommand{\gen}{\textsf{Gen}}
\newcommand{\enc}{\textsf{Enc}}
\newcommand{\dec}{\textsf{Dec}}
\newcommand{\mac}{\textsf{Mac}}
\newcommand{\sign}{\textsf{Sign}}
\newcommand{\verfy}{\textsf{Verfy}}
\newcommand{\negl}{\textup{negl}}

% quantum computing
% gates
\newcommand{\cnot}{\textup{\textsc{cnot}}}
\newcommand{\hdm}{\textup{\textsc{h}}}
\newcommand{\tphase}{\textup{\textsc{t}}}
\newcommand{\cphase}{\textup{\textsc{cphase}}}
\newcommand{\swap}{\textup{\textsc{swap}}}
\newcommand{\negate}{\textup{\textsc{not}}}
\newcommand{\QFT}{\textup{QFT}}

% Boolean Functions
\newcommand{\MAJ}{\textup{\textsc{maj}}}
\newcommand{\NOT}{\textup{\textsc{not}}}
\newcommand{\OR}{\textup{\textsc{or}}}
\newcommand{\AND}{\textup{\textsc{and}}}
\newcommand{\NAND}{\textup{\textsc{nand}}}
\newcommand{\EQ}{\textup{\textsc{eq}}}
\newcommand{\IP}{\textup{\textsc{ip}}}
\newcommand{\DISJ}{\textup{\textsc{disj}}}
\newcommand{\Parity}{\textup{\textsc{parity}}}
\newcommand{\Threshold}{\textup{\textsc{thr}}}

\newcommand{\GS}{\textup{\textsc{gs}}}
\newcommand{\dejo}{\textup{\textsc{DeJo}}}
\newcommand{\STAB}{\textup{\textsc{stab}}}

% algorithms
\newcommand{\algo}{\mathcal{A}}
\newcommand{\maxcut}{\textup{\textsc{MaxCut}}}
\newcommand{\sat}{\textup{\textsc{sat}}}
\newcommand{\partition}{\textup{\textsc{Partition}}}
\newcommand{\bosonsample}{\textup{\textsc{BosonSampling}}}

% complexity measures
\newcommand{\vcdim}{\mathsf{VCdim}}
\DeclareMathOperator{\certificate}{\mathsf{Cert}}
\DeclareMathOperator{\s}{\mathsf{s}}
\DeclareMathOperator{\bs}{\mathsf{bs}}
\DeclareMathOperator{\adeg}{\mathsf{\widetilde{deg}}}
% \DeclareMathOperator{\adv}{\mathsf{Adv}}
\DeclareMathOperator{\dqc}{\mathsf{D}}
\DeclareMathOperator{\rqc}{\mathsf{R}}
\DeclareMathOperator{\qqc}{\mathsf{Q}}
\DeclareMathOperator{\cmc}{\mathsf{C}}
\DeclareMathOperator{\rcmc}{\mathsf{RC}}
\DeclareMathOperator{\qcmc}{\mathsf{QC}}
\let\deg\relax
\DeclareMathOperator{\deg}{\mathsf{deg}}
\DeclareMathOperator{\poly}{\textup{poly}}

% complexity classes
\newcommand{\reduceto}{\le_P}
\let\cclass\textup
\let\P\relax
\newcommand{\P}{\cclass{P}}
\newcommand{\PP}{\cclass{PP}}
\newcommand{\NP}{\cclass{NP}}
\newcommand{\sharpP}{\cclass{\#P}}
\newcommand{\coNP}{\cclass{co-NP}}
\newcommand{\PH}{\cclass{PH}}
\newcommand{\NPC}{\cclass{NPC}}
\newcommand{\BQP}{\cclass{BQP}}
\newcommand{\QMA}{\cclass{QMA}}
\newcommand{\PSPACE}{\cclass{PSPACE}}
\newcommand{\BPP}{\cclass{BPP}}

% Optimization
\newcommand{\subjectto}{\textup{subject to  }}

\let\iff\relax
\newcommand{\iff}{\text{iff}}
\newcommand{\eff}{\textup{eff}}
\newcommand{\st}{\text{ s.t. }}
\newcommand{\otherwise}{\text{otherwise}}
\newcommand{\T}{\intercal}
\newcommand{\OPT}{\textup{OPT}}


\newcommand\vartextvisiblespace[1][.5em]{%
  \makebox[#1]{%
    \kern.07em
    \vrule height.3ex
    \hrulefill
    \vrule height.3ex
    \kern.07em
  }% <-- don't forget this one!
}
\newcommand{\visiblespace}{\vartextvisiblespace}

%%%%%%%%%%%%%%%%%%%%%%%%%%%%%%%%%%%%
%%%%%%%%%%%%% Physics %%%%%%%%%%%%%%
%%%%%%%%%%%%%%%%%%%%%%%%%%%%%%%%%%%%
\newcommand{\zpartition}{\mathcal{Z}}
\newcommand{\llaplacian}{\mathfrak{L}}
\newcommand{\dlagrangian}{\mathcal{L}}
\newcommand{\eaction}{\mathcal{A}}
\newcommand{\action}{\mathcal{S}}
\newcommand{\hhat}{\hat{H}}
\newcommand{\xhat}{\hat{x}}
\newcommand{\phat}{\hat{p}}
\newcommand{\qhat}{\hat{q}}
\newcommand{\nhat}{\hat{n}}
\newcommand{\pihat}{\hat{\pi}}
\newcommand{\phihat}{\hat{\phi}}
\newcommand{\oph}{\mathbf{H}}
\newcommand{\opx}{\mathbf{x}}
\newcommand{\opp}{\mathbf{p}}
\newcommand{\opq}{\mathbf{q}}
\newcommand{\vecx}{\vec{x}}
\newcommand{\vecp}{\vec{p}}
\newcommand{\veck}{\vec{k}}
\newcommand{\vecq}{\vec{q}}
\newcommand{\vbk}{\vb{k}}
\newcommand{\vbs}{\vb{s}}
\newcommand{\vbx}{\vb{x}}
\newcommand{\vbn}{\vb{n}}
\newcommand{\vbp}{\vb{p}}
\newcommand{\vbq}{\vb{q}}
\newcommand{\vbr}{\vb{r}}
\newcommand{\vbe}{\vb{e}}
\newcommand{\vbv}{\vb{v}}
\newcommand{\vbw}{\vb{w}}
\newcommand{\vbB}{\vb{B}}
\newcommand{\vbE}{\vb{E}}
% \newcommand{\acreation}{\hat{a}^\dagger}
% \newcommand{\aannihilation}{\hat{a}}
\newcommand{\acreation}{\hat{a}^\dagger}
\newcommand{\aannihilation}{\hat{a}}
\newcommand{\bcreation}{\hat{b}^\dagger}
\newcommand{\bannihilation}{\hat{b}}
\newcommand{\ccreation}{\hat{c}^\dagger}
\newcommand{\cannihilation}{\hat{c}}
\newcommand{\homega}{\hbar \omega}
\newcommand{\opsigma}{\hat{\bm{\sigma}}}
\newcommand{\hatsigma}{\hat{\sigma}}
\newcommand{\bmhsig}{\bm{\hat{\sigma}}}
\newcommand{\hsig}{\hat{\sigma}}
\newcommand{\si}{\hat{\sigma}_0}
\newcommand{\sx}{\hat{\sigma}_x}
\newcommand{\sy}{\hat{\sigma}_y}
\newcommand{\sz}{\hat{\sigma}_z}
\newcommand{\splus}{\hat{\sigma}_+}
\newcommand{\sminus}{\hat{\sigma}_-}
\newcommand{\px}{\hat{X}}
\newcommand{\py}{\hat{Y}}
\newcommand{\pz}{\hat{Z}}
\newcommand{\pI}{\hat{I}}
\newcommand{\schrodinger}{\textup{Schr\"{o}dinger }}
\newcommand{\tc}{T_c}
\newcommand{\alembertian}{\square}
\newcommand{\vecA}{\vb{A}}
\newcommand{\magfield}{\vb{B}}
\newcommand{\elefield}{\vb{E}}

\newcommand{\deltat}{\Delta t}
\newcommand{\deltatau}{\Delta \tau}

%%%%%%%%%%%%%%%%%%%%%%%%%%%%%%%%%%%%
%%%%%%%%%%%%% Quantum Computing %%%%%%%%%%%%%%
%%%%%%%%%%%%%%%%%%%%%%%%%%%%%%%%%%%%
% \newcommand{\gcommutator}[1]{[[ #1 ]]}
\usepackage{stmaryrd}
\newcommand{\gcommutator}[1]{\llbracket #1 \rrbracket}
\renewcommand{\llaplacian}{\hat{\mathfrak{L}}}
%\newcommand{\zpartition}{\mathcal{Z}}
\newcommand{\hamiltonian}{\hat{H}}
\newcommand{\ew}{\hat{W}}
\newcommand{\ob}{\hat{O}}
\newcommand{\U}{\hat{U}}
\newcommand{\dm}{\hat{\rho}}
\newcommand{\oracle}{\hat{O}}
\newcommand{\D}{\mathcal{D}}
\newcommand{\proj}{\hat{P}}
%\newcommand{\deltat}{\Delta t}
%\newcommand{\deltatau}{\Delta \tau}
\newcommand{\cz}{\textup{\textsc{cz}}}
\newcommand{\cx}{\textup{\textsc{cx}}}
\newcommand{\toffoli}{\textup{\textsc{toffoli}}}
\newcommand{\lleft}{\leftarrow}
\newcommand{\rright}{\rightarrow}
\newcommand{\intinf}{\int_{-\infty}^{\infty}}

% disable subsections and subsubsections in the TOC
\makeatletter
%\def\l@subsection#1#2{}
\def\l@subsubsection#1#2{}
\makeatother

\begin{document}
%%%%%%%%%%%%%%%%%%%
\title{Entanglement witness by quantum circuits}
\author{Jue Xu}
\email{juexu@cs.umd.edu}
\author{Qi Zhao}
\email{email}


%\affiliation{Department of Computer Science, University of Maryland, College Park.}
\date{\today}
%%%%%%%%%%%%%%%%%%%
% \vspace{10mm}
\begin{abstract}
	Machine learning algorithms are applied to the entanglement witness problem.
\end{abstract}

\maketitle
% \setcounter{tocdepth}{0}
 \tableofcontents
% \newpage

%%%%%%%%%%%%%%%Content%%%%%%%%%%%%%%%
\section{Introduction}
Entanglement \cite{horodeckiQuantumEntanglement2009} is the key ingredient of quantum computation, quantum communication, and quantum cryptography [ref].

\section{Preliminary}
% \subsection{Related works}
\subsection{Notations}
\subsection{Entanglement witness}
\begin{definition}[Entangled state]
\end{definition}
\begin{definition}[Bipartite state]
\end{definition}
\begin{definition}[Multi-partite state]
\end{definition}
\begin{definition}[Fully separable state]
\end{definition}
\begin{definition}[Genuine entangled state]
\end{definition}
\begin{definition}[Graph state]
\end{definition}
\cite{heinEntanglementGraphStates2006}.
MBQC \cite{briegelMeasurementbasedQuantumComputation2009}

\begin{definition}[Entanglement witness]
	entanglement witness $\ew$
	\begin{equation}
		\Tr(\ew\dm) \ge 0 , \forall \text{ separable };\quad
		\Tr(\ew\dm) < 0 , \text{ for some entangled }
	\end{equation}
\end{definition}
\begin{figure}[!ht]
	\centering
	% \includegraphics[width=1\linewidth]{}
	\caption{(a) entanglement witness, PPT criteria, SVM (kernel). (b) with training data}
\end{figure}
\begin{problem}[Entanglement witness with prior]
	\cite{zhouDetectingMultipartiteEntanglement2019}
\end{problem}

\subsection{Shadow tomography}
Inspired by \cite{aaronsonShadowTomographyQuantum2018}, Huang et. al \cite{huangPredictingManyProperties2020}
\begin{definition}[classical shadow]\label{def:classical_shadow}
\end{definition}
\begin{theorem}
\end{theorem}

\section{Quantum Algorithms}

\subsection{Machine learning assisted}
separability classifier by neural network \cite{luSeparabilityEntanglementClassifierMachine2018}.
rigorous quantum advantage of quantum kernel method in SVM \cite{liuRigorousRobustQuantum2021}.
classical machine learning with \nameref{def:classical_shadow} \cite{huangProvablyEfficientMachine2021}.
\begin{definition}[SVM]
\end{definition}
\begin{definition}[Kernel]
\end{definition}

\subsection{Variational quantum circuits}
ansatz

\subsection{Upper bounds and lower bounds}
quantum advantage
\begin{itemize}
	\item input encoding problem \cite{tangQuantumPrincipalComponent2021}
	\item 
\end{itemize}

\section{Numerical Simulation}
% \input{complexity.tex}
% \input{optimization.tex}

\section{Conclusion and Discussion}
todo
\begin{itemize}
	\item experiment
\end{itemize}

%\subsection*{Acknowledgements}
% \thanks{The author thanks} 
% The author thanks
% TikZiT, QuTip

% %\begin{appendices}
%     %\chapter{}
% %\end{appendices}

% %%%%%%%%%%%%%%%Reference%%%%%%%%%%%%%%%
% % \newpage
% % \printbibliography
\bibliographystyle{apsrev4-2}
%\bibliographystyle{alpha}
\bibliography{ref}

%\begin{widetext}
\onecolumngrid
\appendix
%\input{pathintegral.tex}
%\input{hamiltonian.tex}
%\input{cs.tex}
%% !TEX root = ./ew.tex


\section{Definitions}\label{sec:definitions}
\begin{definition}[density matrix]\label{def:density_matrix}
	pure state $\ket{\psi}$;
	A quantum state $\dm$ is defined to be a positive operator $ \dm \in \text{End}(V )$ with $\Tr( \dm ) = 1$.
	density matrix $\dm$ (trace one, Hermitian, PSD)...
\end{definition}
\begin{definition}[POVM]\label{def:povm}
	A positive-operator valued measurement (POVM) $M$ consists of a set of positive operators that sum to the identity operator $\identity$. 
	When a measurement $M = \qty{ E_1 , \dots , E_k }$ is applied to a quantum state $\dm$, the outcome is $i \in [k]$ with probability $p_i = \tr( \dm E_i )$.
	observables ... $\expectation[x]\equiv\expval{\ob_x}:=\tr(\ob_x\dm)$
\end{definition}
\begin{definition}[positive, semidefinite]\label{def:psd}
	denoted $X \preceq Y $ provided $Y-X $ is positive
\end{definition}
\begin{definition}[partial trace]\label{def:partial_trace}
	% partial trace;
	reduced density matrix $\dm_A = \Tr_B(\dm_{AB})$
\end{definition}
% \begin{definition}[partial trace]
% 	partial trace
% \end{definition}
\begin{remark}
	% (def partial trace ...)
	a pure (bipartite) state is entangled $\iff$ the reduced state $\dm^A=\Tr_A(\dm)$ is mixed.
	The mixedness of this reduced state allows one to quantify the amount of entanglement in this state.
\end{remark}
\begin{definition}[partial transpose]\label{def:partial_transpose}
	\cite{horodeckiSeparabilityMixedStates1996}
	The partial transpose (PT) operation - acting on subsystem $A$ - is defined as
	\begin{equation}
		\op{k_A,k_B}{l_A,l_B}^{\T_A}
		:= \op{l_A,k_B}{k_A,l_B}
	\end{equation}
	where $\qty{\ket{k_A,k_B}}$ is a product basis of the joint system AB.
\end{definition}
\begin{definition}[maximally entangled]
	a state vector is \emph{maximually entangled} $\iff$ the reduced state at one qubit is maximally mixed, i.e.,
	$\Tr_A(\op{\psi})=\frac{1}{2}$.
\end{definition}
\begin{definition}[Schmidt measure]\label{def:schmidt_measure}
	Consider the following bipartite pure state, written in Schmidt form:
	\begin{equation}
		\ket{\psi} = \sum_i^r \sqrt{\lambda_i} \ket{\phi_i^A}\otimes\ket{\phi_i^B}
		\label{eq:schmidt_decomposition}
	\end{equation}
	where $\qty{\ket{\phi_i^A}}$ is a basis for $\hilbertspace_A$ and $\qty{\ket{\phi_i^A}}$ for $\hilbertspace_B$.
	The strictly positive values $\sqrt{\lambda_i}$ in the Schmidt decomposition are its \emph{Schmidt coefficients}. 
	The number of Schmidt coefficients, counted with multiplicity, is called its \emph{Schmidt rank}, or Schmidt number. (Schmidt rank ?? $\text{SR}^A(\psi)=\rank(\rho_{\psi}^A)$)
	Schmidt measure is minimum of $\log_2 r$ where $r$ is number of terms in an expansion of the state in product basis.
\end{definition}
% \begin{remark}
% 	When a bipartite vector is written in the Schmidt basis, it is very easy to compute the partial trace of either subsystem	
% 	\begin{equation}
% 		\Tr_B(\op{\psi}) = \sum_i p_i \op{\phi_i^A}
% 	\end{equation}
% \end{remark}
\begin{definition}[entropy]\label{def:entropy}
	In quantum mechanics (information), the von Neumann \emph{entropy} of a density matrix is $H_N(\dm): = -\Tr(\dm \log \dm)=-\sum_i\lambda_i\log(\lambda_i)$;
	In classical information (statistical) theory, the Shannon entropy of a probability distribution $P$ is  $H_S(P):= -\sum_i P(x_i) \log P(x_i)$.
	relative entropy (\nameref{def:divergence})
\end{definition}
\begin{definition}[entanglement entropy]\label{def:entanglement_entropy}
	The bipartite \emph{von Neumann entanglement entropy} $S$
	is defined as the von Neumann entropy of either of
	its reduced density matrix $\dm_A$.
	For a pure state $\dm_{AB}=\op{\Psi}{\Psi}_{AB}$,
	it is given by
	\begin{equation}
		E(\Psi_{AB}) 
		= S(\dm_A)
		= -\Tr(\dm_A \log \dm_A)
		= -\Tr(\dm_B \log \dm_B)
		= S(\dm_B)
	\end{equation}
	where $\dm_A= \Tr_B(\dm_{AB})$ and $\dm_B = \Tr_A(\dm_{AB})$ 
	are the reduced density matrices for each partition.
	With Schmidt decomposition (\cref{eq:schmidt_decomposition}), the entropy of entanglement is simply $-\sum_ip_i^2\log(p_i)$.
	the $n$th Renyi entropy,
	$S_n = \frac{1}{n-1} \log (R_n)$
	% \begin{equation}
	% 	S_n = \frac{1}{n-1} \log (R_n)
	% \end{equation}
	where $R_n = \Tr(\dm^n_A)$
\end{definition}
\begin{example}
	The \nameref{def:schmidt_measure} for any multi-partite \nameref{exm:ghz} states is 1, because there are just two terms.
	Schmidt measure for 1D, 2D, 3D-\nameref{def:cluster_state} is $\floor{\frac{N}{2}}$.
	Schmidt measure of tree is the size of its minimal vertex cover[??].
\end{example}

% \subsubsection{Distance measures}\label{sec:distance_measure}
\begin{definition}[fidelity]\label{def:fidelity}
	Given a pair of states (target $\dm$ and prepared $\dm'$), 
	Uhlmann fidelity $F(\dm,\dm') : = \Tr(\sqrt{\sqrt{\dm}\dm'\sqrt{\dm}})\equiv\norm{\sqrt{\dm}\sqrt{\dm'}}_1$, where $\sqrt{\dm}$ dentoes the positive semidefinite square root of the operator $\dm$. (infidelity $1-F(\dm,\dm')$)
	For any mixed state $\rho$ and pure state $\ket{\psi}$, $F(\dm,\op{\psi})=\sqrt{\mel{\psi}{\dm}{\psi}}\equiv \sqrt{\Tr(\dm\op{\psi})}$ which can be obtained by the Swap-test[?].
	linear fidelity or overlap $F(\dm,\dm'):=\tr(\dm\dm')$.
	% \begin{equation}
	% 	F(\ket{\psi},\ket{\psi'}) :=
	% \end{equation}
	% \begin{equation}
	% 	F(\rho,\rho') : = \Tr \sqrt{\sqrt{\rho}\rho'\sqrt{\rho}}
	% \end{equation}
\end{definition}
different distance measures \cite{badescuQuantumStateCertification2017}
\begin{notation}[norm]\label{def:norm}
	Schatten p-norm $\norm{x}_p:= (\sum_i \abs{x_i}^p)^{1/p}$.
	Euclidean norm $l_2$ norm;
	Spectral (operator) norm $\norm{\vbx}_{\infty}$;
	Trace norm $\norm{A}_{\Tr}\equiv\norm{A}_{1}:=\Tr(\abs{A})\equiv\Tr(\sqrt{A^\dagger A})$, $\abs{A}:=\sqrt{A^\dagger A}$, $p=1$;
	Frobenius norm $\norm{A}_{F}:=\sqrt{\Tr(A^\dagger A)}$, $p=2$;
	Hilbert-Schmidt norm $\norm{A}_{HS}:=\sqrt{\sum_{i,j} A_{ij}^2 }=?\sum_{i\in I}\norm{Ae_i}_H^2$;
	Hilbert-Schmidt inner product $\expval{A,B}_{\textup{HS}}:=\Tr(A^\dagger B)$,
	Frobenius inner product $\expval{A,B}_{\textup{F}}:=\Tr(A^\dagger B)$?
	(in finite-dimensionala Euclidean space, the HS norm is identical to the Frobenius norm)
	Although the Hilbert-Schmidt distance is arguably not too meaningful, operationally, one can use Cauchy-Schwarz to relate it to the very natural trace distance. 
	shadow norm ...
\end{notation}
\begin{definition}[distance]\label{def:distance}
	For mixed states, trace distance $d_{\tr}(\dm,\dm') : = \frac{1}{2} \norm{\dm-\dm'}_1$.
	For pure states, $d_{\tr}(\ket{\psi},\ket{\psi'}) : = \frac{1}{2}\norm{\op{\psi} -\op{\psi'}}_1 = \sqrt{1-\abs{\ip{\psi}{\psi'}}^2}$.
	% \begin{equation}
	% 	d_{tr}(\rho,\rho') : = \frac{1}{2} \norm{\rho-\rho'}_1
	% \end{equation}
	fidelity and trace distance are related by the inequalities
	\begin{equation}
		1-F\le D_{\tr}(\dm,\dm') \le \sqrt{1-F^2}
	\end{equation}
	variation distance of two distribution $d_{var}(p,p') : = \frac{1}{2} \sum_i \abs{p_i-p_i'} = \frac{1}{2} \norm{p-p'}_1$.
	% It also has an operational meaning: it is the greatest probability with which one can discriminate a draw from p and a draw from q.
	$l_2$ distance ... Hellinger distance ... HS distance $D_{\text{HS}}(\dm,\dm'): = \norm{\dm-\dm'}_{\text{HS}}=\sqrt{\Tr((\dm-\dm')^2)}$
	% As a probability metric, the $l_2$ distance is somewhat unnatural. For example, it does not satisfy the “data processing inequality”, meaning that there is a stochastic operation that increases $l_2$ distance. However it is by far the easiest distance to calculate, as it is a simple polynomial in p and q; further, it can be related to the total variation distance
\end{definition}


% \subsection{Stabilizer formalism}\label{sec:stabilizer_formalism}
denote a group by $\group$ and a subgroup $\subgroup$. 
\begin{definition}[Pauli group]
\end{definition}
\begin{definition}[Clifford group]\label{def:clifford}
\end{definition}
\begin{definition}[Stabilizer]\label{def:stabilizer}
	An observable $S_k$ is a stabilizing operator of an $n$-qubit state $\ket{\psi}$ if the state $\ket{\psi}$ is an eigenstate of $S_k$ with eigenvalue 1,

	A stabilizer set $S = \qty{ S_1, \dots , S_n}$ consisting of n mutually commuting and independent stabilizer operators is called the set of stabilizer “generators”.
\end{definition}
Many highly entangled $n$-qubit states can be uniquely defined by $n$ stabilizing operators which are locally measurable, i.e., they are products of Pauli matrices.
A stabilizer $S_i$ is an $n$-fold tensor product of $n$ operators chosen from the one qubit Pauli operators $\qty{\identity,X,Y,Z}$.
% \begin{remark}
	An $n$-partite(qubit) graph state can also be uniquely determined by $n$ independent stabilizers, 
	$S_i:= X_i \bigotimes_{j\in n}Z_j$, 
	which commute with each other and $\forall i,S_i\ket{G}=\ket{G}$.??
	The graph state is the unique eigenstate wtih eignevalue of +1 for all the $n$ stabilizers.
	As a result, a graph state can be written as a product of stailizer projectors, $\op{G}=\prod_{i=1}^n \frac{S_i +\identity}{2}$.
	\nameref{def:stabilizer} formalism
	% stabilizer formalism
% \end{remark}
\begin{example}[GHZ]\label{exm:ghz}
	% Greenberger-Horne-Zeilinger (GHZ) 
	For GHZ state: $\ket{\ghz}:=\frac{1}{\sqrt{2}}(\ket{0}^{\otimes n} + \ket{1}^{\otimes n} )$,
	the projector based witness 
	% $\ew_{\ghz_3}=\identity/2 - \op{\ghz}$ 
	\begin{equation}
		\ew_{\ghz_3}= \frac{1}{2} \identity - \op{\ghz}
	\end{equation}
	requires four measurement settings.
	% (the direction projection is hard to evaluate because the state is entangled).
	For three-qubit GHZ state 
	\cite{tothDetectingGenuineMultipartite2005},
	the local measurement witness
	\begin{equation}
		\ew_{\ghz_3} := \frac{3}{2} \identity - \px^{(1)}\px^{(2)}\px^{(3)}
		- \frac{1}{2} \qty(
			\pz^{(1)} \pz^{(2)} + 
			\pz^{(2)} \pz^{(3)} + 
			\pz^{(1)} \pz^{(3)} 
		)
	\end{equation}
	This witness requires the measurement of the $\qty{\sx^{(1)},\sx^{(2)},\sx^{(3)}}$ and $\qty{\sz^{(1)},\sz^{(2)},\sz^{(3)}}$ settings.
	For $n$-qubit case, detect genuine $n$-qubit entanglement close to $\ghz_n$
	\begin{equation}
		\ew_{\ghz_n} = (n-1) \identity - \sum_{k=1}^n S_k{(\ghz_n)}
	\end{equation}
	where $\stbz_k$ is the \nameref{def:stabilizer} ... \cite{tothEntanglementDetectionStabilizer2005}
	Detecting Genuine Multipartite Entanglement with Two Local Measurements \cite{tothDetectingGenuineMultipartite2005}
\end{example}
% \begin{question}
% 	how far/close to the target state (entanglement witness), noise limit?
% \end{question}

\nameref{def:graph_state} is an important large class of multipartite states in quantum information,
because (connected) graph states represent a large class of genuine multipartite entangled states that have concise representations.
Typical graph states include cluster states, \nameref{exm:ghz} states, and the states involved in error correction (toric code).
	% Any connected graph state is \nameref{def:fully_entangled} state.
It worth noting that 2D cluster (rectangular lattice graph) state is the universal resource for the measurement based quantum computation (MBQC) \cite{briegelMeasurementbasedQuantumComputation2009}.
% cluster state is the special case of graph state.
% \begin{definition}[cluster state]\label{def:cluster_state}
% \end{definition}
% \begin{example}[graph states]
	% (line, ring; 
	% hypercube, Petersen graph; 
	% cluster state in two dimensinos, which corresponds to a rectangular lattice.)
	% The \nameref{exm:ghz} state corresponds to the star graph and the complete gtaph (\cref{fig:graph_state}). 
	% This is easily seen by applying Hadamard unitaries $\U_\hdm^{V\backslash a}$ to all but one qubit $a$ in the GHZ-state, 
	% which yields the star graph state with $a$ as the central qubit.
	% The Petersen graph is not LC-equivalent to its isomorphism (exchanging the labels at each ed of the five ``spokes"). However, the lists of Schmidt ranks (or, equivalently, the connectivity functions) of these graphs coincide.
	% The class of CSS (error correction) states corresponds to the class of 2-\nameref{exm:colorable} graphs.
	The entanglement in a graph state is related to the topology of its underlying graph \cite{heinEntanglementGraphStates2006}.
% \end{example}
\begin{remark}
	LU, LC equivalence, local operations and classical communication (LOCC), 
\end{remark}
\begin{definition}[graph state]\label{def:graph_state}
	Given a simple graph (undirected, unweighted, no loop and multiple edge) $G=(V,E)$, a graph state is constructed as 
	from the initial state $\ket{+}^{\otimes n}$ corresponding to $n$ vertices.
	Then, apply controlled-Z gate to every edge, that is 
	$\ket{G}:=\prod_{(i,j)\in E}\textsf{cZ}_{(i,j)} \ket{+}^{\otimes n}$
	with $\ket{+}: = (\ket{0}+\ket{1})/\sqrt{2}$.
	% \begin{equation}
	% 	\ket{G}:=\prod_{(i,j)\in E}\textsf{cZ}_{(i,j)} \ket{+}^{\otimes n}
	% 	,\;\ket{+}: = (\ket{0}+\ket{1})/\sqrt{2} .
	% \end{equation}
	% \begin{itemize}
	% 	\item vertices: $\ket{+}^{\otimes n}$
	% 	\item edges: apply controlled-Z to every edge,
	% 	that is $\ket{G}=\prod_{(i,j)\in E}\textsf{cZ}_{(i,j)} \ket{+}^{\otimes n}$
	% \end{itemize}
	% cluster states are defined on lattice (graphs).
\end{definition}
\begin{question}
	how to relate graph state entanglement to \nameref{prm:graph_property_test}
	..??. (graph kernel??)
	\cite{heinEntanglementGraphStates2006}.
	witness; bounds; \nameref{def:graph_property}? vertex cover?
	Hamiltonian cycle of a graph state? 
\end{question} 

% \begin{proposition}\label{thm:entanglement_witness_gme}
% 	Given a state $\ket{\psi}$,
% 	the \nameref{def:entanglement_witness} operator $\ew_{\psi}$ can witness \nameref{def:gme}  near $\ket{\psi}$ with $c=5/8$ in \cref{eq:entanglement_witness}
% 	% \nameref{def:genuinely_entangled}
% 	% \begin{equation}
% 	% 	\ew_{\psi} = \frac{5}{8}\identity - \op{\psi} 
% 	% \end{equation}
% 	that is, $\expval{\ew_{\psi}}\ge 0$ for any separable state in $S_b$.
% \end{proposition}
% If the \nameref{def:fidelity} (quantum kernel? similarity) of the prepared state $\dm_{\text{pre}}$ with the target state $\ket{\psi}$, i.e., $\Tr(\dm_{pre}\op{\psi})$, exceeds $5/8$, $\dm_{pre}$ possesses GME.
% It is generally difficult to evaluate the quantity $\Tr(\dm_{pre}\op{\psi})$ by the direct projection on $\ket{\psi}$, as it is an entangled state.
\begin{table}[!ht]
\centering
\begin{tabular}{c|c|c|c|c|c|c|c|c}
	 & $\ket{\ghz_3}$ & $\ket{W_3}$ & $\ket{CL_3}$ & $\ket{\psi_2}$ & $\ket{\D_{2,4}}$ & $\ket{\ghz_n}$ & $\ket{W_3}$ & $\ket{G_n}$ \\
	\hline
	$\alpha$ & $1/2$ & $2/3$ & $1/2$ & $3/4$ & $2/3$ & $1/2$ & $(n-1)/n$ & $1/2$ \\
	maximal $p_{\noise}$ & 4/7 & 8/21 & 8/15 & 4/15 & 16/45 & $1/2 \cdot (1-1/2^n)^{-1}$ & & \\
	local measurements & 4 & 5 & 9 & 15 & 21 & $n+1$ & $2n-1$ & depend on graphs \\
	\hline
\end{tabular}
\caption{\cite{guhneEntanglementDetection2009}}
\end{table}

\section{Machine learning background}
% As these areas are extremely broad, we cannot completely review all known literature; we will simply give pointers to some of the best known and most relevant results.

% In this work, we restrict ourself to supervised learning (mainly SVM), where we are given a set of labeled data for training to predict labels of new data.

Notations:
The (classical) training data (for supervised learning) is a set of $m$ data points $\qty{(\vbx^{(i)}, y^{(i)})}^{m}_{i=1}$ 
where each data point is a pair $(\vbx,y)$.
Normally, the input (e.g., an image) $\vbx:= (x_1,x_2,\dots,x_d) \in \realnumber^d$  is a vector where $d$ is the number of \emph{features}
and its \emph{label} $y\in\Sigma$ is a scalar with some discrete set $\Sigma$ of alphabet/categories. 
For simplicity and the purpose of this paper, we assume $\Sigma=\qty{-1,1}$ (binary classification).


\subsection{Support vector machine}\label{sec:svm}
SVM is a typical supervised learning algorithm for classification. Taking the example of classifying cat/dog images, supervised learning means we are given a dataset in which every image is labeled either a cat or a dog such that we can find a function classifying new images with high accuracy. More precisely,  the training dataset is a set of pairs of features X and their labels y. In the image classification case, features are obtained by transforming all pixels of an image into a vector. In SVM, we want to find a linear function, that is a hyperplane which separates cat data from dog data. So, the prediction label is given by the sign of the inner product (projection) of the hyperplane and the feature vector. We can observe that the problem setting of image classification by SVM is quite analogous to entanglement detection, where input data are quantum states now and the labels are either entangled or separable.


\begin{definition}[SVM]\label{def:svm}
	Given a set of (binary) labeled data,
	support vector machine (SVM) is designed to
	find a hyperplane (a linear function) such that maximize the margin between two partitions...
	\begin{equation}
		\max_{\vb{w}}
		\norm{\vb{w}}^2
		\text{ s.t. }
		\forall i , y^{(i)}\cdot (\vb{w}\cdot\vb{x}+b)\ge 1.
	\end{equation}
\end{definition}
Lagrange multipliers $\alpha$
\begin{equation}
	L = \frac{1}{2}\norm{\vbw}^2 - \sum_i^m \alpha^{(i)} \qty(\vbw\cdot \vbx^{(i)} + b) + \sum_i^m \alpha^{(i)}
\end{equation}
% dual probelm with $\vbw = \sum_m \alpha^{(m)} y^{(m)} \vbx^{m}$
% \begin{equation}
% 	L_D = \sum_i \alpha^{(i)} - \frac{1}{2} \sum_{ij} \alpha^{(i)} \alpha^{(j)} y^{(i)} y^{(i)}
% 	\vbx^{(i)} \cdot \vbx^{(j)}
% \end{equation}

\subsubsection{kernel method}
However, note that SVM is only a linear classifier. while most real-world data, such as cat/dog images and entangled/separable quantum states are not linearly separable. For example, with this two dimension dataset, we are unable to find a hyperplane to separate red points from the purple points very well. Fortunately, there is a very useful tool called kernel method or kernel trick to remedy this drawback. The main idea is mapping the features to a higher dimensional space such that  they can be linearly separated in the high dimensional feature space. Just like this example, two dimensional data are mapped to the three dimensional space. Now, we can easily find the separating plane. With SVM and kernel methods, we expect to find a generic and flexible way for entanglement detection.
\nameref{def:kernel}
\begin{definition}[kernel]\label{def:kernel}
	In general, the kernel function $\kernel:\mathcal{X}\times \mathcal{X} \to \realnumber$ measures the similarity between two input data points by an inner product
	\begin{equation}
		\kernel (\vbx,\vbx') : = \expval{\phi(\vbx),\phi(\vbx')}
	\end{equation}
	If the input $\vbx\in \realnumber^d$ (conventional machine learning task, e.g., image classification), the feature map $\phi(\vbx): \realnumber^d\to \realnumber^n$ ($d < n$) from a low dimensional space to a higher dimensional space.
	The corresponding kernel (Gram) matrix $\mathbf{K}$ should be a positive, semidefinite (PSD) matrix, i.e. all eigenvalues are non-negative
\end{definition}
\begin{example}[kernels]
	Some common kernels: 
	the polynomial kernel $\kernel_{\text{poly}}(\vbx,\vbx') := (1+\vbx\cdot\vbx')^q$ with feature map $\phi(\vbx)$ ...
	The Gaussian kernel
	$\kernel_{\text{gaus}}(\vbx,\vbx') := \exp(-\gamma\norm{\vbx-\vbx'}^2_2)$ 
		% \begin{equation}
		% 	\kernel_{\text{gaus}}(\vbx,\vbx') := \exp(-\norm{\vbx-\vbx'}^2/(2\beta))
		% \end{equation}
	with an infinite dimensional feature map $\phi(\vbx)$.
	An important feature of kernel method is that kernels can be computed efficiently without evaluating feature map (might be infinite dimension) explicitly.
	% \begin{itemize}
	% 	\item \emph{Gaussian kernel}; 
	% 	\begin{equation}
	% 		\kernel_{\text{gaus}}(\vbx,\vbx') := \exp(-\norm{\vbx-\vbx'}^2/(2\beta))
	% 	\end{equation}
	% 	note that infinite dimensional feature map

		% \item \emph{graph kernel} \cite{kriegeSurveyGraphKernels2020}: given a pair of graphs $(\graph,\graph')$
		% \begin{equation}
		% 	\kernel (\graph,\graph')  =
		% \end{equation}
		% quantum graph kernel $\kernel (\graph,\graph')  = \abs{\ip{\graph}{\graph'}}^2$ ??
		% \cite{baiQuantumJensenShannon2015}

		% \item quantum kernel (transition amplitude / quantum propagator);
		% \begin{equation}
		% 	k_Q(\rho,\rho') := \abs{\ip{\phi(x)}{\phi(x')}}^2 =\abs{\mel{0}{\U_{\phi(x)}^\dagger \U_{\phi(x')} }{0}}^2 = \Tr(\rho\rho')
		% \end{equation}
		% with quantum feature map $\phi(x): \mathcal{X}\to \op{\phi(x)}$

		% \item \emph{shadow kernel}:
		% given two density matrices $\rho$ and $\rho'$
		% \begin{equation}
		% 	k_{\shadow}(\rho,\rho') := 
		% \end{equation}

	% 	\item neural tagent kernel \cite{jacotNeuralTangentKernel2020}: proved to be equivalent to deep neural network \cite{gaoEfficientRepresentationQuantum2017}
	% \end{itemize}
\end{example}

similarity measures? advantages? why? (isomorphism?)
\begin{definition}[divergence]\label{def:divergence}
	KL divergence (relative \nameref{def:entropy}): measure the distance (similarity) between two probability distributions:
	\begin{equation}
		\kl (P || Q) := \sum P(x) \log (P(x)/Q(x))
	\end{equation}
	symmetric version: Jensen-Shannon divergence (machine learning)
	\begin{equation}
		\jsd (P || P') := \frac{1}{2} \qty(\kl(P|| M) + \kl(P'||M))
		\equiv H_S(M)-\frac{1}{2} (H_S(P) - H_S(P') ) 
	\end{equation}
	where $M=(P+P')/2$ and Shannon \nameref{def:entropy} $H_S$.
	Analogously, quantum Jensen-Shannon divergence $D_{\qjs}$ of two density matrices can be defined...
	\begin{equation}
		D_{\qjs}(\dm||\dm'):= 
		H_V(\dm_M) - \frac{1}{2} (H_V(\dm) - H_V(\dm') ) 
	\end{equation}
	as a quantum graph kernel ($\dm$ induced by quantum random walk)
\end{definition}
\begin{definition}[geometric difference]\label{def:geometric_difference}
	\begin{equation}
		g(K^1|| k^2) = \sqrt{\norm{\sqrt{K^2} (K^1)^{-1}\sqrt{K^2}}_{\infty}}
	\end{equation}
	where $\norm{\cdot}_{\infty}$ is the spectral \nameref{def:norm}.
\end{definition}

\subsubsection{Graph kernel}
\begin{definition}[graph property]\label{def:graph_property}
	% The setting of graph property testing provides a natural class of partial graph properties.
	monotone ...
\end{definition}
\begin{example}[colorable]\label{exm:colorable}
	$k$-colorable is a graph property, i.e., allow for a coloring of the vertices with $k$ colors such that no two adjacent vertices have the same color.
	A graph is bipartite $\iff$ 2-colorable.
	other graph properties: isomorphism; vertex cover; Hamiltonian cycle ...
\end{example}
\begin{problem}[graph property test]\label{prm:graph_property_test}
	\textbf{promise}: the input graph either has a property, or is $\epsilon$-far from having the property, meaning that we must change at least an $\epsilon$ fraction of the edges to make the property hold.
\end{problem}
\begin{theorem}[bounds for graph property test]
\end{theorem}
\begin{question}
	\cite{montanaroSurveyQuantumProperty2018}
	Is there any graph property which admits an exponential quantum speed-up?
	\cite{ben-davidSymmetriesGraphProperties2020}
	depends on input model (query adjacency matrix/list)
	% quantum algorithms (bounds) for graph properties \cite{ben-davidSymmetriesGraphProperties2020}
\end{question}
Graphs is another kind of data which is fundamentally different from a real value vector because of vertex-edge relation and graph isomorphism.
So, graph kernel \cite{kriegeSurveyGraphKernels2020} need additional attention.
\begin{definition}[graph kernel]\label{def:graph_kernel}
	given a pair of graphs $(\graph,\graph')$,
	\emph{graph kernel} is $\kernel (\graph,\graph')  =$.
	% \begin{equation}
	% \end{equation}
	quantum graph kernel $\kernel (\graph,\graph')  = \abs{\ip{\graph}{\graph'}}^2$ ??
	\cite{baiQuantumJensenShannon2015}	
\end{definition}

\subsubsection{Quantum kernel}
related works:
\begin{itemize}
	\item quantum kernel method: estimate kernels by quantum algorithms (circuits)	\cite{havlicekSupervisedLearningQuantum2019}
	\cite{schuldQuantumMachineLearning2019}: for classical problem (data)

	\item rigorous and robust quantum advantage of quantum kernel method in SVM \cite{liuRigorousRobustQuantum2021}. group structured data \cite{glickCovariantQuantumKernels2021}

	\item power of data in quantum machine learning \cite{huangPowerDataQuantum2021}: input??? projected quantum kernel
\end{itemize}
% \begin{remark}[\cite{horodeckiDirectDetectionQuantum2002}]
% 	Direct entanglement detections, can be \textbf{employed as sub-routines in quantum computation}. For example, one may consider performing or not performing a quantum operation on a given quantum system conditioned on some part of quantum data being entangled or not. In fact direct entanglement detections can be viewed as quantum computations solving an inherently quantum decision problem: given as an input n copies of decide whether is entangled. Here the \textbf{input data is quantum} and such a decision problem cannot even be even formulated for classical computers. 
% 	% Nonetheless the problem is perfectly well defined for quantum computers.

% 	% For the sake of completeness we should also mention here that there are two-particle observables, called entanglement witnesses which can detect quantum entanglement is some special cases (see [20,8]). They have positive mean values on all separable states and negative on some entangled states. Therefore \textbf{any individual entanglement witness leaves many entangled states undetected}. When is unknown we need to check infinitely many witnesses, which effectively reduces this approach to the quantum state estimation. However, let us point out any witness defines a positive map which can be used in our test.
% \end{remark}
\begin{definition}[quantum kernel]\label{def:quantum_kernel}
	quantum kernel 
	with quantum feature map $\phi(\vbx): \mathcal{X}\to \op{\phi(\vbx)}$
	\begin{equation}
		k_Q(\rho,\rho') := \abs{\ip{\phi(\vbx)}{\phi(\vbx')}}^2 =\abs{\mel{0}{\U_{\phi(\vbx)}^\dagger \U_{\phi(\vbx')} }{0}}^2 =? \Tr(\rho\rho') \equiv \expval{\dm,\dm'}_{\textup{HS}}
	\end{equation}
	where $\U_{\phi(\vbx)}$ is a quantum circuit or physics process that encoding an input $\vbx$.
	In quantum physics, quantum kernel is also known as transition amplitude (quantum propagator);
\end{definition}
\begin{theorem}
	On quantum computers, evaluating the trace distances is probably hard since even judging whether $\dm$ and $\dm'$ have large or small trace distance is known to be QSZK-complete \cite{watrousQuantumComputationalComplexity2008}, where QSZK (quantum statistical zero-knowledge) is a complexity class that includes BQP (bounded-error quantum polynomial time).
	% Variational Quantum Algorithms for Trace Distance and Fidelity Estimation
\cite{chenVariationalQuantumAlgorithms2022}
\end{theorem}
% \begin{definition}[multivariate trace estimation]\label{def:multivariate_trace_estimation}
% 	The task of estimating quantities like 
% 	\begin{equation}
% 		\Tr(\rho_1 \cdots \rho_m)
% 		\tag{multivariate traces}
% 	\end{equation}
% 	given access to copies of the quantum states $\rho_1$  through $\rho_m$.
% \end{definition}
% is a fundamental building block in quantum information science

% power of data - 
\begin{proposition}[\cite{huangPowerDataQuantum2021}]
	If a classical algorithm without training data can compute (label) $y=f(x)=\mel{x}{\U_{\textup{QNN}}^\dagger \ob U_{\textup{QNN}}}{x}$ (with amplitude encoding) efficiently (poly time in ...) for any $\U_{\textup{QNN}}$ and $\ob$, then $\nameref{def:bpp}=\nameref{def:bqp}$ (which is believed unlikely).
\end{proposition}
\begin{proposition}[\cite{huangPowerDataQuantum2021}]
	Training an arbitrarily deep quantum neural network $\U_{\qnn}$ with a trainable observable $\ob$ is equivalent to training a \nameref{def:quantum_kernel} method with kernel $k_{Q}(\vbx,\vbx')=\Tr(\dm(\vbx)\dm'(\vbx'))$
\end{proposition}
\begin{definition}[projected quantum kernel]\label{def:projected_quantum_kernel}
	....
\end{definition}

\subsection{Neural network}\label{sec:neural_network}
\subsubsection{neural network and kernel}
\begin{definition}[neural tagent kernel]\label{def:neural_tangent_kernel}
	neural tangent kernel \cite{jacotNeuralTangentKernel2020}: proved to be equivalent to deep neural network \cite{gaoEfficientRepresentationQuantum2017} in the limit ...
	\begin{equation}
		k_{\ntk} \qty(S_T(\dm_l),\tilde{S}_T(\dm_{l'}))
		=
		\expval{
			\phi^{(\ntk)}(S_T(\dm_l)),
			\phi^{(\ntk)}(\tilde{S}_T(\dm_l))
		}
	\end{equation}
\end{definition}



\subsubsection{quantum neural network}\label{sec:quantum_neural_network}
% \subsection{Unsupervised: PCA}

\section{Hardness assumptions}
\begin{definition}[\NP]\label{def:np}
	\NP, \NP-hard, \NP-complete
\end{definition}
\begin{definition}[\sharpP]\label{def:sharpp}
	\sharpP
\end{definition}
\begin{definition}[QMA]\label{def:qma}
	QMA
\end{definition}
\begin{definition}[\BPP]\label{def:bpp}
	\BPP
\end{definition}
\begin{definition}[\BQP]\label{def:bqp}
	\BQP
\end{definition}
%\end{widetext}

\end{document}